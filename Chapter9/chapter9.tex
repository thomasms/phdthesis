%chapter 9
\chapter{Conclusions and Outlook}
This thesis has been comprised of two independent sections with one underlying and unifying theme, neutral particle detection in noble gases. In the first two chapters we introduced how neutral particles, particularly neutrinos, neutrons and photons can be detected by means of noble gases and discussed the benefits of using them while also exploring the technical feats that must be overcome to make them viable detection mediums. 

Chapters 3 to 5 presented the LAGUNA-LBNO project, proposed as the next generation long baseline neutrino experiment in Europe, with the focus on the near detector. The design, simulation and evaluation of a newly proposed near detector design was performed using a bespoke  software framework encompassing several Monte Carlo packages. With this being the initial study of the near detector, previous studies on such a detector did not exist, and due to this a lot of the preliminary and base work was covered within this thesis. Loosely based on the T2K ND280 detector, we were able to use similar parameters to help in the design. From the results of several Monte Carlo studies we have managed to quantify the basic detector capabilities and set the path for future detectors of similar design. We have shown for a TPC of 2 $\times$ 2 $\times$ 2 m$^{3}$ at 20 bar the expected number of neutrino interactions is 0.1785 $\pm$ 0.0003 p.p.p for the 400 GeV in positive focusing. Considering only $\nu_{\mu}$ with energies of 10 GeV and less, within a fiducial volume at 1.8 $\times$ 1.8 $\times$ 1.8 m$^{3}$, this is reduced to 0.0628 $\pm$ 0.0002 p.p.p for the same beam. CCQE interactions are then 0.00779 $\pm$ 0.00007 $\nu_{\mu}$ p.p.p for the same selection.

Background rates were also heavily examined in the ND design studies, with $\mu$'s the particles of biggest concern we found that 44.5 $\pm$ 0.5 p.p.p for the 400 GeV in positive focusing were reaching/passing the TPC due to interactions from outside the TPC. Although they contribute to only 14.5\% of the total particles entering the TPC from background interactions, they pose the largest threat to pileup with energies similar to that expected from interactions within the TPC. It was shown that 75.6\% of these arise due to rock interactions are therefore hard to minimise, however when compared to the number arising from the beam (punch through muons) their contribution is smaller at 36.7\% compared to the 61.1\% of beam muons at 70 p.p.p. Thus with $\sim$1/2 $\mu$ tracks/ m$^{2}$ / $\mu$s and drift times of $\sim$100 $\mu$s, pile up is unavoidable but manageable assuming the vertex can be resolved to within $\sim$1 $\mu$ tracks/ 700 cm$^{2}$ / spill.

Ultimately a pressurised GAr TPC ND design will not achieve the requirements of the LAGUNA-LBNO proposal using current technologies. With a required signal event normalisation uncertainty of 3\% a first order analysis using only the final state $\mu$ indicates that this is not achievable, estimating an uncertainty at 3.8\%. The difficulty with the presented design is primarily due to the high number of DIS events expected at that TPC, 50.3\% of CC interactions, due to the high energy beam where large amounts of energy are lost to neutrons and $\pi^{0}$s. The main benefit of the GAr medium with its great track and position reconstruction capabilities, due to the high granularity of the detector, is then lost due to the energy lost to neutral particles. 

Overall with the main aim of the project to determine the mass hierarchy of the neutrinos to beyond the 5$\sigma$ threshold, other ND detector designs should be considered if this is to be achieved. However at the time of writing, the continuation of the LAGUNA-LBNO project has been halted due to economical and political reasons, with costs expected to exceed the 100 million EUR estimate. It is important to note that Liverpool University and myself drove the ND design and created initial discussions on its requirements. I played a large role in the design and the study has led to interesting work within the neutrino community, paving the way for other very long baseline neutrino experiments, with LBNX becoming the focus of future neutrino experiments. The software framework created largely by myself can then be used for these future designs and take the information gained from this investigation to design an improved ND for high energy, broad spectrum beams.

Chapters 6 to 9 presented the MODES-SNM project which discussed a new and novel way of detecting nuclear radiation, neutrons and gammas, in a portable manner. From the conceptual design of using pressurised $^{4}$He gas, instead of the conventionally used $^{3}$He, right through to the construction and testing of a fully working prototype, we have shown that the MODES-SNM system can be used in the combat against radionuclide trafficking. 

We have shown that both gamma and neutron radiation sources can be detected with successful identification of the gamma sources: $^{60}$Co, $^{241}$Am, and $^{133}$Ba while in both stationary and motion detection modes. Other gamma sources have been identified in laboratory conditions with AmBe, Pu and $^{252}$Cf neutron sources being detected successfully even upon shielding. Real life demonstrations have also showed promising results with many end users interested in the potential of the MODES-SNM prototype. 

We have presented a new analysis to be used to discriminate between neutrons and $\gamma$'s showing that even in high levels of gamma contamination (up to 76\%) we can achieve higher than required detection efficiencies, exceeding 96\%, while maintaining reasonable false alarm rates (< 1 per hour).

The MODES-SNM prototype has paved the way to becoming a new commercial system that could be a very competitive alternative to current systems, with it showing lower false alarm rates with cheaper costs due to the rising costs of $^{3}$He. The MODES-SNM collaboration has been largely successful and a great achievement considering we managed to deliver a working prototype on time and to budget.