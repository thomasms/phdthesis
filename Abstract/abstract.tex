% Thesis Abstract -----------------------------------------------------
\begin{abstract}
Neutral particles, particularly neutrons, gammas and neutrinos are difficult to detect and measure due to their lack of electric charge. Noble fluids are a powerful medium when detecting such particles due to their ability to collect charge and scintillation light. The LAGUNA-LBNO and MODES-SNM projects are two independent projects that focus on using this concept to detect neutral particles of interest. The two projects are consecutively discussed in this thesis.
%LAGUNA-LBNO is a feasibility study to asses the potential physics reach of a next generation long baseline neutrino experiment with a potential baseline spanning over 2300 km. The primary aim of the experiment is probe the nature of the neutrino by determination of its mass hierarchy and examine CP violation in the lepton sector. The MODES-SNM project is the complete study, from design to prototyping of a system used to detect special nuclear materials using pressurised $^{4}$He and Xenon detectors. The two projects are consecutively discussed in this thesis.
% at 180 bar and 50 bar respectively. 

A study on a potential near detector design to be used within the proposed LAGUNA-LBNO experiment is presented. We introduce a novel design for the near detector based on a pressurised gas Argon TPC at 20 bar surrounded by layers of plastic scintillator, encompassed in a pressurised gas chamber. Monte Carlo studies form the basis of the study with focus on detector interaction rates and assessment of the basic detector properties and parametrisation. Based on a 2 $\times$ 2 $\times$ 2 $\times$ m$^{3}$ TPC we estimate 0.1785 $\pm$ 0.0003 (stat) $\nu$ p.p.p interactions for a 400 GeV neutrino beam in positive focusing and 0.0628 $\pm$ 0.0002 (stat) $\nu_{\mu}$ p.p.p interactions for energies 0-10 GeV in a 1.8 $\times$ 1.8 $\times$ 1.8 $\times$ m$^{3}$ fiducial volume. Conversely we can expect high muon backgrounds in the TPC at 44.5 $\pm$ 0.5 $\mu$ (stat) p.p.p, arising from neutrino interactions with external detector components (non TPC) and surrounding rock interactions. With the inclusion of the muons arising from the beam directly at 70 p.p.p (estimated) we can expect $\sim$1-2 $\mu$ tracks in the TPC / 700 cm$^{2}$ / spill.

Within the MODES-SNM section of the thesis a prototype system is designed, tested and analysed using $^{4}$He, instead of the commonly favourable $^{3}$He, for fast neutron detection. A neutron-gamma discrimination analysis is performed based on a pulse shape discrimination technique from the collected scintillation light of PuBe and $^{60}$Co sources in laboratory conditions. For high levels of gamma contamination (up to 76\%) detection efficiencies exceeding 96\% can be achieved with the prototype system while maintaining reasonable false alarm rates (< 1 per hour).

\end{abstract}
% ----------------------------------------------------------------------

